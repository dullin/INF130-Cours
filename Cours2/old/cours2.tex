\documentclass[aspectratio=169,usenames,dvipsnames]{beamer}
\usepackage{listings}

\usetheme[progressbar=frametitle]{metropolis}

\makeatletter
\setlength{\metropolis@titleseparator@linewidth}{2pt}
\setlength{\metropolis@progressonsectionpage@linewidth}{2pt}
\setlength{\metropolis@progressinheadfoot@linewidth}{2pt}
\makeatother

\lstset{language=[Visual]Basic, basicstyle=\footnotesize\ttfamily, keywordstyle=\color{blue}, commentstyle=\color{OliveGreen},frame=single}

\title{INF130 : Ordinateurs et programmation}
\subtitle{Semaine 2 – Présentation, Constantes, Boucles, opérateurs logiques, sous-programmes}
\author{Hugo Leblanc}

\begin{document}
    \maketitle

    %--- Next Frame ---%
    \begin{frame}{Sommaire}
        \tableofcontents
    \end{frame}
    %--- Next Frame ---%

    \section{Présentation}
    \subsection{Présentation et commentaires}
    \begin{frame}{Présentation et commentaires}
        \begin{itemize}
            \item Commentaires
            \begin{itemize}
                \item En-tête
                \item Actions
            \end{itemize}
            \item Identation
            \item Espacement
            \item Nom significatifs
        \end{itemize}
    \end{frame}
    %--- Next Frame ---%
    \subsection{Constantes}
    \begin{frame}[fragile]{Constantes}
        \begin{itemize}
            \item Les constantes remplacent l’utilisation de valeurs numériques statiques dans le code.
            \item Par convention, elles sont écrites en majuscules au début du module (en dehors des procédures).
            \item On assigne directement la valeur à la constante durant la déclaration.
        \end{itemize}
        \begin{lstlisting}
Const MA_CONSTANTE As Integer = 50
        \end{lstlisting}
    \end{frame}
    %--- Next Frame ---%
    \section{Structure de contrôle répétitives}
    \begin{frame}[fragile]{Boucles - While}
        \begin{itemize}
            \item Les boucles permettent de répéter un bloc de code.
            \item La boucle while répète le bloc de code tant que la condition de répétition est remplis
            \item Le while est utilisé quand on ne connait pas le nombre d’itérations de boucle à faire.
        \end{itemize}
        \begin{lstlisting}
While x < 10
    'Instructions
Wend
        \end{lstlisting}
    \end{frame}
    %--- Next Frame ---%
    \begin{frame}{Exercices}
        \begin{itemize}
            \item Saisit un nombre à l'utilisateur et recommence la saisit tant que le nombre n'est pas 0.
            \item Calcule la somme des nombres de 1 a n. On saisit n.
            \item Procédure qui saisit un nombre à l'utilisateur et va afficher le nombre factoriel de la saisit $(5! = 1x2x3x4x5)$

        \end{itemize}
    \end{frame}
    %--- Next Frame ---%
    \subsection{Boucles - For}
    \begin{frame}[fragile]{Boucles - For}
        \begin{itemize}
            \item Lorsqu’on connait le nombre d’itérations de la boucles, on utilise la boucle for qui intègre un compteur dans sa configuration.
        \end{itemize}
\noindent\begin{minipage}{.25\textwidth}
\begin{lstlisting}
For i = 1 To 10
    MsgBox i
Next i
\end{lstlisting}
\end{minipage}\hfill
\begin{minipage}{.32\textwidth}
\begin{lstlisting}
For i = 1 To 10 Step 2
    MsgBox i
Next i
\end{lstlisting}
\end{minipage}\hfill
\begin{minipage}{.32\textwidth}
\begin{lstlisting}
For i = 5 To 1 Step -1
    MsgBox i
Next i
\end{lstlisting}
\end{minipage}
    \end{frame}
    %--- Next Frame ---%
    \begin{frame}{Exercices}
        \begin{itemize}
            \item Calcule la somme des nombre de 1 à n. On saisit n. Utilisez un for.
            \item Compte le nombre de 0 entrez au clavier sur 10 essais
\end{itemize}
    \end{frame}
    %--- Next Frame ---%
    \subsection{Opérateurs logiques}
    \begin{frame}[fragile]{Opérateurs logiques}
        \begin{itemize}
            \item Les opérateurs logiques opèrent sur des valeurs booléennes
            \item La conjonction ET : And
            \item La disjonction OU : Or
            \item La négation NON : Not
        \end{itemize}
    \end{frame}
    %--- Next Frame ---%
    \begin{frame}[fragile]{Exercice}
        \begin{itemize}
            \item Procédure qui saisit l'âge de l'utilisateur et qui indique si ce dernier a droit au tarif réduit. Le tarif réduit est disponible pour les personnes d’âge mineur (<18) ou d’âge d'or (>60).
        \end{itemize}
    \end{frame}
    %--- Next Frame ---%
    \section{Sous-programmes}
    \begin{frame}[fragile]{Sous-programmes}
        \begin{itemize}
            \item Un sous-programme est un bloc de code réalisant une tâche précise.
            \item La fonction retourne (se transformer en) une valeur.
            \item Un sous-programme qui ne retourne rien est nommée une procédure.
            \item Un sous-programme peut avoir des paramètres. Ceux-ci dictent l'utilisation du sous-programme.
        \begin{lstlisting}
variable = NomFonction(paramètre1,paramètre2)
        \end{lstlisting}
        \end{itemize}
    \end{frame}
    %--- Next Frame ---%
\end{document}