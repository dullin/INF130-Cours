\documentclass[aspectratio=169,usenames,dvipsnames]{beamer}
\usepackage{listings}

\usetheme[progressbar=frametitle]{metropolis}

\makeatletter
\setlength{\metropolis@titleseparator@linewidth}{2pt}
\setlength{\metropolis@progressonsectionpage@linewidth}{2pt}
\setlength{\metropolis@progressinheadfoot@linewidth}{2pt}
\makeatother

\lstset{language=[Visual]Basic, basicstyle=\footnotesize\ttfamily, keywordstyle=\color{blue}, commentstyle=\color{OliveGreen},frame=single}

\title{INF130 : Ordinateurs et programmation}
\subtitle{Semaine 4 – Module, références, validation et tests}
\author{Hugo Leblanc}

\begin{document}
    \maketitle

    %--- Next Frame ---%
    \begin{frame}{Sommaire}
        \tableofcontents
    \end{frame}
    %--- Next Frame ---%

    \section{Sous-programmes}
    \subsection{Durée de vie des variables}
    \begin{frame}[fragile]{Durée de vie des variables}
        \begin{itemize}
            \item Tout ce qui se passe à l'intérieur des fonctions est détruit après l'appel de la fonction.
            \item Toutes déclaration de variables à l'intérieur d'une fonction est détruite après l'appel de la fonction.
            \item Seul le retour est renvoyé.
            \item Le mot clé \alert{Static} permet de conserver un variable vivante entre ses appels de sous-programmes.
        \end{itemize}
    \end{frame}
    %--- Next Frame ---%
    \subsection{Portées des éléments}
    \begin{frame}[t]{Portées des éléments}
        \begin{itemize}
            \item La portée d’un élément indique sa visibilité par rapport aux autres modules.
            \item Si un élément est visible, il peut être appelé (utilisé) à l’intérieur de sa portée.
            \item La portée d’une procédure inclue ses déclarations de variables avec \alert{Dim}.
            \item La portée privée (\alert{Private}) d’un module inclue ses déclaration de variables et de sous-programme privées.
            \item La portée publique (\alert{Public}) d’un module inclue toutes les variables et sous-programmes publiques d’un projet (classeur).
            \item Les variables de portée globale sont \alert{INTERDITES} dans le cours, sauf avis contraire.
        \end{itemize}
    \end{frame}
        %--- Next Frame ---%
    \subsection{Passage par référence - ByRef}
    \begin{frame}{Passage par référence - ByRef}
        \begin{itemize}
            \item Le passage par référence reçoit une variable à la place d’une valeur comme dans le passage par valeur.
            \item La référence est donc lié entre l’appellant du sous-programme et le sous-programme lui-même.
            \item Une modification d’une variable passé en référence restera modifié après l’exécution du sous-programme.
            \item Les noms entre le paramètre passée et le nom du paramètre n’a pas d’importance.
        \end{itemize}
    \end{frame}
    %--- Next Frame ---%
    \subsection{Validation}
    \begin{frame}{Validation}
        \begin{itemize}
            \item La validation permet de regarder et valider des informations avant de continuer dans un programme
            \item La validation est surtout utilisée quand nous recevons des informations de l’extérieur (avec InputBox)
            \item La validation permet de générer une erreur avant d’avoir des résultats erronés. 
            \item Le mot-clé \alert{End} permet de terminer l'exécution de notre programme prématurément.
        \end{itemize}
    \end{frame}
    %--- Next Frame ---%
    \subsection{Test}
    \begin{frame}{Test}
        \begin{itemize}
            \item Le test d'une fonction permet de s'assurer de son comportement avant de passer à la création d'élément plus complexes.
            \item Les tests peuvent être dynamique avec des saisis et affichages ou statique par rapport à des valeurs et réponse déjà connus.
            \item Pour de grand projet, les tests statiques sont utilisé pour s'assurer du bon fonctionnement au fur et à mesure de la conception.
        \end{itemize}
    \end{frame}
    %--- Next Frame ---%
    \begin{frame}{Exercices}
        \begin{itemize}
            \item Écrivez une fonction qui retourne le nombre de fois que la fonction à été appellé. Utilisez le mot-clé Static.
            \item Écrivez une procédure qui reçoit une variable par référence et un nombre. La procédure additionne le nombre à la variable donnée.
            \item Écrivez une procédure qui reçoit une valeur entière et arrète l'exécution du programme si celle-ci n'est pas positive.
        \end{itemize}
    \end{frame}
    %--- Next Frame ---%
\end{document}