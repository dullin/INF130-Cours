\documentclass[aspectratio=169,usenames,dvipsnames]{beamer}
\usepackage{listings}

\usetheme[progressbar=frametitle]{metropolis}

\makeatletter
\setlength{\metropolis@titleseparator@linewidth}{2pt}
\setlength{\metropolis@progressonsectionpage@linewidth}{2pt}
\setlength{\metropolis@progressinheadfoot@linewidth}{2pt}
\makeatother

\lstset{language=[Visual]Basic, basicstyle=\footnotesize\ttfamily, keywordstyle=\color{blue}, commentstyle=\color{OliveGreen},frame=single}

\title{INF130 : Ordinateurs et programmation}
\subtitle{Semaine 8 – Types }
\author{Hugo Leblanc}

\begin{document}
    \maketitle

    %--- Next Frame ---%
    \begin{frame}{Sommaire}
        \tableofcontents
    \end{frame}
    %--- Next Frame ---%

    \section{Type définis}
    \subsection{Description}
    \begin{frame}{Description}
        \begin{itemize}
            \item Un type défini est une nouvelle structure de données (comme les tableaux ou les chaines de caractères).
            \item Sous un seul identificateur, des champs (avec des noms de notre propre crus) vont chacun contenir des valeurs.
        \end{itemize}
    \end{frame}
    %--- Next Frame ---%
    \subsection{Création}
    \begin{frame}[fragile]{Création de types définis}
        \begin{itemize}
            \item La déclaration d’un type défini se fait à l’extérieur des sous-programmes. Le type sera disponible pour tous les sous-programmes.
            \item Chaque champs aura un type spécifique.
            \item Les champs peuvent aussi être des tableaux.      
        \end{itemize}
\begin{lstlisting}
Type mon_type
    champ1 As Integer
    champ2() As Double
    champ3 As String
End Type
\end{lstlisting} 
    \end{frame}
    %--- Next Frame ---%
    \subsection{Utilisation des types définis}
    \begin{frame}[fragile]{Utilisation des types définis}
        \begin{itemize}
            \item On utilise un type défini comme un autre type connu (Integer, String, etc.).
            \item On doit le déclaré pour en faire sont utilisation.
            \item L’accès au champs se fait avec l’opérateur point « . ».
        \end{itemize}
\begin{lstlisting}
Sub test_type()
    Dim type1 as mon_type

    type1.champ1 = 5
    ReDim type1.champ2(5)
    type1.champ2(4) = 4.5
    type1.champ3 = "allo"
End Sub
\end{lstlisting}  
        
    \end{frame}
    %--- Next Frame ---%
    
\end{document}