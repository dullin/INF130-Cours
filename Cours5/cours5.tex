\documentclass[aspectratio=169,usenames,dvipsnames]{beamer}
\usepackage{listings}

\usetheme[progressbar=frametitle]{metropolis}

\makeatletter
\setlength{\metropolis@titleseparator@linewidth}{2pt}
\setlength{\metropolis@progressonsectionpage@linewidth}{2pt}
\setlength{\metropolis@progressinheadfoot@linewidth}{2pt}
\makeatother

\lstset{language=[Visual]Basic, basicstyle=\footnotesize\ttfamily, keywordstyle=\color{blue}, commentstyle=\color{OliveGreen},frame=single}

\title{INF130 : Ordinateurs et programmation}
\subtitle{Semaine 5 – Chaine de caractères}
\author{Hugo Leblanc}

\begin{document}
    \maketitle

    %--- Next Frame ---%
    \begin{frame}{Sommaire}
        \tableofcontents
    \end{frame}
    %--- Next Frame ---%

    \section{Chaine de caractères}
    \subsection{Table ASCII}
    \begin{frame}{Table ASCII}
        \begin{itemize}
            \item Chaques caractères utillisées dans une chaine est représenté dans la table de conversion ASCII.
            \item Deux fonctions de VBA nous permets de faire les conversion d’une à l’autre:
            \begin{itemize}
                \item Chr retourne le caractères associé à un code ASCII.
                \item Asc retourne le code ASCII associé à un caractère donné.
            \end{itemize} 
        \end{itemize}
    \end{frame}
    %--- Next Frame ---%
    \begin{frame}{Exercices}
        \begin{itemize}
            \item Écrivez une fonction qui convertit un caractère représentant une lettre minuscule en lettre majuscule.
        \end{itemize}
    \end{frame}
    %--- Next Frame ---%
    \subsection{Opérateurs sur le chaines}
    \begin{frame}[t]{Opérateurs sur les chaines}
        \begin{itemize}
            \item \alert{=} permet l’assignation d’une chaine
            \item Opérateurs de comparaisons permettent de faire une comparaison par rapport à la valeur de la table ASCII entre deux chaines.
            \item \alert{\&} permet la concaténation entre deux valeurs qui donne une chaine
            \item \alert{+} permet la concaténation qui donne une valeur numérique si une des deux expressions est numérique.
        \end{itemize}
    \end{frame}
    %--- Next Frame ---%
    \begin{frame}{Exercices}
        \begin{itemize}
            \item Écrivez une fonction qui retourne True si un caractère représente un chiffre et False dans tous les autres cas.
            \item Écrivez une fonction qui retourne True si un caractère représente une lettre majuscule et False dans tous les autres cas.
            \item Écrivez une fonction qui retourne True si un caractère représente une lettre minuscule et False dans tous les autres cas.
            \item Écrivez une fonction qui retourne True si un caractère représente une lettre et False dans tous les autres cas.
            \item Écrivez une fonction nommée lpad qui reçoit une chaîne de caractères et ajoute n blancs au début de la chaîne.
                    À titre d'exemple, l'appel lpad("allo", 3) retourne \"   allo\".

        \end{itemize}
    \end{frame}
    %--- Next Frame ---%
    \subsection{Fonctions sur les chaines}
    \begin{frame}{Fonctions sur les chaines}
        \begin{itemize}
            \item \alert{Len} – Donne le nombres de caractères dans la chaine.
            \item \alert{UCase} / \alert{LCase} – Convertie en majuscules ou minuscule un chaine.
            \item \alert{Left} / \alert{Right} – Extrait une partie de la chaine à partir de la gauche ou la droite.
            \item \alert{Mid} – Extrait une partie de la chaine à partir d’un caractère définie.
            \item \alert{Trim} – Enlève les blancs à gauche et à droite de la chaine.
            \item \alert{InStr} – Cherche une chaine dans une autre.

        \end{itemize}
    \end{frame}
    %--- Next Frame ---%
    \begin{frame}{Exercices}
        \begin{itemize}
            \item Écrivez une fonction nommée ltrim qui reçoit une chaîne de caractères et retourne celle-ci sans les blancs se trouvant au début.
            \item Écrivez une fonction qui reçoit une chaîne de caractères et un caractère. Elle retourne le nombre d'occurrences de ce caractère dans la chaîne (le nombre de fois que ce caractère se retrouve dans la chaîne).
            \item Écrivez une fonction qui reçoit deux chaînes de caractères; la première contient une phrase alors que la seconde contient une liste de caractères à conserver. La fonction parcourt la phrase et à chaque fois qu'elle trouve un caractère qui n'est pas dans la seconde chaîne et qui n'est pas un blanc, elle le remplace par une étoile. Elle retourne la phrase obtenue.
                    À titre d'exemple, l'appel phrase\_censuree("vive le vent", \"eit \") retourne la chaîne \"*i*e *e *e*t\".

        \end{itemize}
    \end{frame}
    %--- Next Frame ---%
\end{document}