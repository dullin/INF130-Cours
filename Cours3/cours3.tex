\documentclass[aspectratio=169,usenames,dvipsnames]{beamer}
\usepackage{listings}

\usetheme[progressbar=frametitle]{metropolis}

\makeatletter
\setlength{\metropolis@titleseparator@linewidth}{2pt}
\setlength{\metropolis@progressonsectionpage@linewidth}{2pt}
\setlength{\metropolis@progressinheadfoot@linewidth}{2pt}
\makeatother

\lstset{language=[Visual]Basic, basicstyle=\footnotesize\ttfamily, keywordstyle=\color{blue}, commentstyle=\color{OliveGreen},frame=single}

\title{INF130 : Ordinateurs et programmation}
\subtitle{Semaine 3 – Création de sous-programmes}
\author{Hugo Leblanc}

\begin{document}
    \maketitle

    %--- Next Frame ---%
    \begin{frame}{Sommaire}
        \tableofcontents
    \end{frame}
    %--- Next Frame ---%

    \section{Sous-programmes}
    \subsection{Sous-programmes}
    \begin{frame}{Sous-programmes}
        \begin{itemize}
            \item Un sous-programme est un bloc de code réalisant une tâche précise. 
            \item Il existe deux type de sous-programmes:           
            \begin{itemize}
                \item Prodédures (Sub)
                \item Fonctions (Function)
            \end{itemize}
            \item La différence entre une procédure et une fonction est qu'une fonction retourne une valeur tandis qu'une procédure ne retourne aucune valeur.
        \end{itemize}
    \end{frame}
    %--- Next Frame ---%
    \subsection{Procédures}
    \begin{frame}{Procédures}
        \begin{itemize}
            \item La procédure permet d’executé une série d’instruction répondant à une tache précise.
            \item On peut appeler (exécuter) une procédure déjà existante avec l’utilisation d’un Call suivit du nom de la procédure.
            \item La procédures peut recevoir des paramètres d’entrées. Nous allons voir plus loin comment créer des sous-programmes avec des paramètres.
        \end{itemize}
    \end{frame}
    %--- Next Frame ---%
    \subsection{Fonctions}
    \begin{frame}[fragile]{Fonctions}
        \begin{itemize}
            \item Les fonctions retourne une valeurs après son exécution.
            \item Le type de valeur de retour doit être défini à l’énoncé de la fonction.
            \item La valeur de retour doit être assigné durant l’exécution de la fonction.
            \item La nom de la valeur de retour est le nom de la fonction.
            \item Puisque la fonction retourne une valeur, elle peut être utilisé durant une expression.
        \end{itemize}
\noindent\begin{minipage}{.48\textwidth}
\begin{lstlisting}
Function retourne_5() As Integer
    retourne_5 = 5
End Function
\end{lstlisting}
\end{minipage}\hfill
\begin{minipage}{.48\textwidth}
\begin{lstlisting}
Sub test_fcn()
    Dim x As Integer
    x = retourne_5() + 8
    Call MsgBox(x)
End Sub
\end{lstlisting}
\end{minipage}\hfill
    \end{frame}
    %--- Next Frame ---%
    \begin{frame}{Exercices}
        \begin{itemize}
            \item Écrivez une fonction qui trouve l’aire d’un triangle à partir de sa base et sa hauteur. Saisir la base et la hauteur.
            \item Écrivez une fonction qui détermine si un nombre est impaire. \\Indice : utilisez Mod. Saisir le nombre à tester.
        \end{itemize}
    \end{frame}
    %--- Next Frame ---%
    \subsection{Paramètres}
    \begin{frame}[fragile]{Paramètres}
        \begin{itemize}
            \item Les sous-programmes peuvent recevoir des paramètres d’entrées.
            \item Les paramètres sont les informations critiques dont les sous-programmes ont besoins pour bien fonctionner.
            \item Chaque paramètre est typé et devient une variable utilisable durant l’exécution du sous-programmes.
        \end{itemize}
\begin{lstlisting}
Sub affiche_param(ByVal entree As Integer)
    Call MsgBox(entree)
End Sub
\end{lstlisting}
\begin{lstlisting}
Function retourne_param(ByVal entree As Integer) As Integer
    retourne_param = entree
End Function
\end{lstlisting}

    \end{frame}
    %--- Next Frame ---%
    \begin{frame}{Exercices}
        \begin{itemize}
            \item Écrivez une fonction qui trouve l’aire d’un triangle à partir de sa base et sa hauteur.
            \item Écrivez une fonction qui détermine si un nombre est impaire. \\Indice : utilisez Mod.
\end{itemize}
    \end{frame}
    %--- Next Frame ---%
    \section{Règles des sous-programmes}
    \subsection{Durée de vie des variables}
    \begin{frame}[fragile]{Durée de vie des variables}
        \begin{itemize}
            \item Tout ce qui se passe à l'intérieur des fonctions est détruit après l'appel de la fonction.
            \item Toutes déclaration de variables à l'intérieur d'une fonction est détruite après l'appel de la fonction.
            \item Seul le retour est renvoyé.
        \end{itemize}
    \end{frame}
    %--- Next Frame ---%
    \subsection{Passage par valeur - ByVal}
    \begin{frame}{Passage par valeur - ByVal}
        \begin{itemize}
            \item Les paramètres et les retours sont renommé pour la durée du sous-programme.
            \item Seul leur valeurs seront transféré entre le sous-programme et l'appelant.
            \item Les noms des paramètres n'ont aucune incidence.
            \item L'ordre des paramètre est ce qui sera considéré.
        \end{itemize}
    \end{frame}
    %--- Next Frame ---%
\end{document}