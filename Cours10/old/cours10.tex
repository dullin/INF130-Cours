\documentclass[aspectratio=169,usenames,dvipsnames]{beamer}
\usepackage{listings}

\usetheme[progressbar=frametitle]{metropolis}

\makeatletter
\setlength{\metropolis@titleseparator@linewidth}{2pt}
\setlength{\metropolis@progressonsectionpage@linewidth}{2pt}
\setlength{\metropolis@progressinheadfoot@linewidth}{2pt}
\makeatother

\lstset{language=[Visual]Basic, basicstyle=\footnotesize\ttfamily, keywordstyle=\color{blue}, commentstyle=\color{OliveGreen},frame=single}

\title{INF130 : Ordinateurs et programmation}
\subtitle{Semaine 10 – Formulaires}
\author{Hugo Leblanc}

\begin{document}
    \maketitle

    %--- Next Frame ---%
    \begin{frame}{Sommaire}
        \tableofcontents
    \end{frame}
    %--- Next Frame ---%

    \section{Événements d'Excel}
    \begin{frame}{Événements d'Excel}
        \begin{itemize}
            \item Les objets d’Excel peuvent aussi avoir des événements reliés avec du code.
            \item Un événement est une action pris par l’utilisateur ou l’application qui est reconnu par l’objet. (Ex: Sélectionné une feuille, Ouvrir Excel, Clicker sur un bouton)
            \item La liste d’événement des différents objets est disponible dans des menus déroulant dans l’éditeur VBA.
        \end{itemize}
    \end{frame}
    %--- Next Frame ---%
    \section{Formulaires}
    \begin{frame}{Formulaires}
        \begin{itemize}
            \item Un formulaire est créé au même niveau qu’un module.
            \item Le formulaire aura deux aspect: visuel et code.
            \item On peut modifier l’aspect visuel pour y ajouter des contrôles de notre choix.
            \item On peut lié les contrôles à du code avec les événements des objets.
        \end{itemize}
    \end{frame}
    %--- Next Frame ---%
    \begin{frame}{Gestion du formulaires}
        \begin{itemize}
            \item Chaque contrôles nous donne un nouvel objet avec des événements.
            \item On doit géré les événements dans la partie code du formulaire.
            \item On utilize les methodes show et hide de l’objet du formulaire pour faire afficher et disparaitre le formulaire à partir de nos module.
            \item Les formulaires sont une exception à la règle habituelle sur les variables globales. Nous les utiliserons pour garder en mémoire l’information contenu dans le formulaire.
        \end{itemize}
    \end{frame}
\end{document}