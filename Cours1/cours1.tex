\documentclass[aspectratio=169,usenames,dvipsnames]{beamer}
\usepackage{listings}

\usetheme[progressbar=frametitle]{metropolis}

\makeatletter
\setlength{\metropolis@titleseparator@linewidth}{2pt}
\setlength{\metropolis@progressonsectionpage@linewidth}{2pt}
\setlength{\metropolis@progressinheadfoot@linewidth}{2pt}
\makeatother

\lstset{language=[Visual]Basic, basicstyle=\footnotesize\ttfamily, keywordstyle=\color{blue}, commentstyle=\color{OliveGreen},frame=single}

\title{INF130 : Ordinateurs et programmation}
\subtitle{Semaine 1 – Programmation, variables, entrées/sorties, conditionnelles}
\author{Hugo Leblanc}

\begin{document}
    \maketitle
    \begin{frame}[shrink=10]{Sommaire}
        \smallskip
        \tableofcontents
    \end{frame}
    %--- Next Frame ---%

    \section{Présentation du cours}
    \subsection{Présentation personnelle}
    \begin{frame}{Présentation personnelle}
        \begin{description}
            \item Hugo Leblanc
            \item hugo.leblanc@etsmtl.ca
            \item Baccalauréat en génie électrique de l’ÉTS en 2012
            \item 14 ans de programmation scolarisée
            \item Spécialisation en systèmes embarqués
        \end{description}
    \end{frame}
    %--- Next Frame ---%
    \subsection{{Présentation du cours}}
    \begin{frame}[t]{Présentation du cours}
        \begin{itemize}
            \item Plan de cours
            \begin{itemize}
                \item Pondération
                \item Date de remise
                \item Politique de plagiat
            \end{itemize}
            \pause
            \item Moodle ena.etsmtl.ca
            \begin{itemize}
                \item Centre de toures les intéractions du cours
                \item Notes de cours, exerciecs, remise, messages
            \end{itemize}
            \pause
            \item Structure de travail du cours
            \begin{itemize}
                \item 1000 erreurs à faire avant la fin de la session
                \item Très facile de perdre le contrôle de la matière
                \item Les travaux pratiques sont des préparations aux examens
            \end{itemize}
        \end{itemize}
    \end{frame}
    %--- Next Frame ---%
    \section{Cours}
    \begin{frame}{Objectifs de la semaine}
        \begin{itemize}
            \item Éditeur VBA
            \item Modules
            \item Commentaires
            \item Procédures
            \item Variables, types et assignation
            \item Saisie et affichage
            \item Opérateurs
            \item Structure de contrôles conditionnelles (if)
        \end{itemize}
    \end{frame}
    %--- Next Frame ---%
    \subsection{Éditeur VBA}
    \begin{frame}{Éditeur VBA}
        \begin{itemize}
            \item Chaque application de la suite Microsoft Office contient un éditeur VBA qui nous permet de faire de la programmation à l’intérieur de l’application.
            \item Dans le cours, nous utiliserons Excel comme application de base pour tous nos travaux.
            \item Pour ouvrir l’éditeur, appuyer sur \alert{Alt-F11} à l’intérieur d’Excel
            \item Un fichier contenant des programmes est un fichier avec des macros et doit avoir l’extension correspondante à la sauvegarde (\alert{.xlsm})
            \item Un seul éditeur peux traiter plusieurs document Excel en même temps
        \end{itemize}
    \end{frame}
    %--- Next Frame ---%
    \subsection{Modules}
    \begin{frame}{Modules}
        \begin{itemize}
            \item Un module est un regroupement de sous-programme
            \item Plusieurs modules peuvent être dans le même fichier Excel
            \item Les modules sont utilisés pour découper un grand programme en plus petites parties indépendantes
            \item On ajoute un nouveau module par le menu \alert{Insert -> Module}
        \end{itemize}
    \end{frame}
    %--- Next Frame ---%
    \subsection{Commentaires}
    \begin{frame}{Commentaires}
        \begin{itemize}
            \item Le caractère \alert{‘} (apostrophe) précède tous commentaires
            \item Le reste de la ligne après le ‘ ne sera pas considéré par VBA durant l’exécution de code
            \item Le commentaires sont primordiaux à la programmation
            \item Les commentaires sont utilisés en en-tête de modules, fonction et programme ainsi qu’à l’intérieur d’un programme pour aider à comprendre l’intention des instructions.
        \end{itemize}
    \end{frame}
    %--- Next Frame ---%
    \subsection{Procédures}
    \begin{frame}[fragile]{Procédures (sous-programmes)}
        \begin{itemize}
            \item Un sous-programmes est un amalgam d’instructions qui seront exécutées ensemble.
            \item Le premier type de sous programme que nous allons voir est la procédure.
            \item On exécute une procédure en appuyant sur le bouton « play » ou avec le raccourci F5.
        \end{itemize}
        \begin{lstlisting}
'Les parties en bleu sont
'essentielles à la syntaxes.
'Le nom de la procédure est
'à la discretion du programmeur.
Sub NomDeProcedure()
    'Instructions
End Sub
        \end{lstlisting}
    \end{frame}
    %--- Next Frame ---%
    \subsection{Variables, types, Assignation}
    \begin{frame}[fragile]{Variables - Déclaration}
        \begin{itemize}
            \item Une variable est la combinaison d’un espace mémoire réservé, un identificateur, une valeur et un type.
            \item Une variable doit être déclaré avant de pouvoir être utilisé.
            \item La déclaration est sous la forme suivante:
            \begin{lstlisting}
Dim nomVariable As Integer
            \end{lstlisting}
            \item Dim est le mot réservé pour la déclaration d’espace.
            \item Le nom de la variable est à votre discretion.
            \item Le dernier mot est le type de la variable.
            \item Pour nous aider avec les déclarations nous utiliserons la configuration Option Explicit au début de tout nos modules.
            \item Les déclarations se font au début des sous-programme.
        \end{itemize} 
    \end{frame}
    %--- Next Frame ---%
    \begin{frame}{Variables - Types}
        \begin{itemize}
            \item Un variable doit être définie par un type
            \item Le type indique quel genre de donnée peut exister dans la variable
            \item Les types de base sont:
            \begin{itemize}
                \item \alert{Integer} : Les nombres entiers de -32 768 à 32 767.
                \item \alert{Long} : Les nombres entiers de -2 147 483 648 à 2 147 483 647.
                \item \alert{Double} : Les nombres réels (avec une certaine marge d'erreur) de $-1.79769313486232 x 10^308$ à $1.79769313486232 x 10^308$.
                \item \alert{String} : Les chaines de caractères, donc du texte. Par exemple "Allo!" ou "Miam, miam, miam. Les bons gros légumes." . Les chaines de caractères sont toujours délimiter de double guillmets.
                \item \alert{Boolean} : Vrai ou faux (True et False sous VBA).
            \end{itemize}
        \end{itemize}
    \end{frame}
    %--- Next Frame ---%
    \begin{frame}[fragile]{Variables - Assignation}
        \begin{itemize}
            \item On assigne une valeur à une variable l’opérateur =
            \begin{lstlisting}
Dim nomVariable As Integer
nomVariable = 10
            \end{lstlisting}
            \item L’assignation va seulement prendre le type de valeur que la variable peut contenir (ne pas mettre du texte dans une variable de type Integer)
            \item Les variables ont une valeur par nulle défault.
            \item On utilise les valeur dans les variables en invoquant leur nom dans des équations
            \begin{lstlisting}
nomVariable = nomVariable + 5
            \end{lstlisting}
        \end{itemize}
    \end{frame}
    %--- Next Frame ---%
    \subsection{Affichage et saisie}
    \begin{frame}[fragile]{Affichage - MsgBox}
        \begin{itemize}
            \item L’affichage se fait avec un appel à MsgBox. Ce sous-programme doit recevoir une chaine de caractères qui sera affiché par la suite.
            \begin{lstlisting}
MsgBox "Allo monde!"
Call MsgBox("Allo monde!")
            \end{lstlisting}
            \item On peut aussi envoyer une valeur numérique qui sera convertie en chaine avant d’être affiché.
            \begin{lstlisting}
MsgBox 345
            \end{lstlisting}
            \item MsgBox est plus versatile que les exemples plus haut mais il faudra attendre un plus grande connaissance des appel de sous-programme avant d’approfondir le sujet.

        \end{itemize}
    \end{frame}
    %--- Next Frame ---%
    \begin{frame}{Exercice – Salaire hebdomadaire}
        \begin{itemize}
            \item Écrivez un sous-programme avec trois variables (taux horaire, nombre d’heures travaillées et salaire hebdomadaire.
            \item Calculez le salaire à partir du taux horaire et du nombre d’heures travaillées. 
            \item Affichez-le salaire dans une boite de dialogue par la suite.
        \end{itemize}
    \end{frame}
    %--- Next Frame ---%
    \begin{frame}[fragile]{Saisie - InputBox}
        \begin{itemize}
            \item La saisie de texte ou de valeur se fait avec un appel au sous-programme InputBox.
            \item On assigne le retour (la valeur saisie) dans un variable lors de l’appel.
            \begin{lstlisting}
Dim reponse As String
reponse = InputBox("Entrez quelque chose:")
            \end{lstlisting}
            \item Le type du retour est une chaine de caractères. Pour saisir un nombre, on convertie la valeur avant de l’assigner.
            \begin{lstlisting}
Dim nombre As Integer
nombre = Val(InputBox("Entrez quelque chose:"))
            \end{lstlisting}
        \end{itemize}
    \end{frame}
    %--- Next Frame ---%
    \begin{frame}{Exercice – Salaire hebdomadaire avec InputBox}
        Ajoutez à l’exercice précédent une saisie du taux horaire et du nombre d’heures travaillées avec des InputBox
    \end{frame}
    %--- Next Frame ---%
    \subsection{Opérateurs}
    \begin{frame}{Opérateurs}
        \begin{itemize}
            \item Les opérateurs arithmétique /, +, -, * et \^{} sont disponibles pour les opérations numérique.
            \item L’opérateur \textbackslash permet de faire la division entière. La division entière coupe la partie fractionnaire).
            \item L’opérateur Mod permet de faire le module d’un nombre. Le module est le restant après un division.
            \item Les opérateurs relationnels <, <=, >, >=, =, <> (différent) permet de comparer deux valeurs numériques.
            \item L’opérateur \& peut faire la concaténation (coller) deux chaines de caractères ensemble.
        \end{itemize}
    \end{frame}
    %--- Next Frame ---%
    \begin{frame}{Exercice – Salaire hebdomadaire avec concaténation}
        Ajoutez à l’exercice précédent un affichage avec une phrase complète dans le MsgBox.
    \end{frame}
    %--- Next Frame ---%
    \subsection{Structure de contrôle conditionnel}
    \begin{frame}[fragile]{Structure de contrôle conditionnelle - if}
        \begin{itemize}
            \item La structure confitionnelle nous permet de prendre des décisions durant l’exécurtion de nos scripts
            \item La décision à prendre doit être fait sur une expression booléenne (vrai ou faux)
            
            \smallskip
\noindent\begin{minipage}{.45\textwidth}
\begin{lstlisting}
If nombre > 10 Then
    MsgBox "Plus que 10!"
End If
\end{lstlisting}
\end{minipage}\hfill
\begin{minipage}{.45\textwidth}
\begin{lstlisting}
If nombre > 10 Then
    MsgBox "Plus que 10!"
ElseIf nombre > 20 Then
    MsgBox "Plus que 20!"
Else
    MsgBox "Moins..."
End If
\end{lstlisting}
\end{minipage}
            \item Un seul bloc d’instruction est exécuté.
            \item Le elseif peut être répété au besoin.
        \end{itemize}
    \end{frame}
    %--- Next Frame ---%
    \begin{frame}{Exercice – Salaire hebdomadaire avec temps supplémentaire}
        Ajoutez à l’exercice précédant un calcul pour le temps supplémentaire.
    \end{frame}
    %--- Next Frame ---%
\end{document}