\documentclass[aspectratio=169,usenames,dvipsnames]{beamer}
\usepackage{listings}

\usetheme[progressbar=frametitle]{metropolis}

\makeatletter
\setlength{\metropolis@titleseparator@linewidth}{2pt}
\setlength{\metropolis@progressonsectionpage@linewidth}{2pt}
\setlength{\metropolis@progressinheadfoot@linewidth}{2pt}
\makeatother

\lstset{language=[Visual]Basic, basicstyle=\footnotesize\ttfamily, keywordstyle=\color{blue}, commentstyle=\color{OliveGreen},frame=single}

\title{INF130 : Ordinateurs et programmation}
\subtitle{Semaine 6 – Tableaux}
\author{Hugo Leblanc}

\begin{document}
    \maketitle

    %--- Next Frame ---%
    \begin{frame}{Sommaire}
        \tableofcontents
    \end{frame}
    %--- Next Frame ---%

    \section{Tableaux}
    \subsection{Description}
    \begin{frame}{Description}
        \begin{itemize}
            \item Les tableaux sont des collections de plusieurs éléments du même type avec un seul identificateur.
            \item Les tableaux nous permettent de travailler avec de grands échantillons de données.
            \item Nous utiliserons la configuration Option Base 1 dans nos module pour que les tableau commence à la case 1 (et non 0).
        \end{itemize}
    \end{frame}
    %--- Next Frame ---%
    \subsection{Création de tableaux}
    \begin{frame}[fragile]{Création de tableaux}
        \begin{itemize}
            \item La déclaration de tableau se fait comme une variable mais en ajoutant des parenthèses après le nom avec le nombre de cases voulus.
\begin{lstlisting}
Dim monTableau(5) As Integer
Dim AutreTableau(1 To 10) As Double
Dim DernierTableau(2 To 4) As String
\end{lstlisting}
            \item La numérotation des cases commence à 1 par défaut (grâce à Option Base 1).
            \item On peut avoir des tableaux de plusieurs dimensions. On délimite chaque dimension avec un virgule.
            \item Par convention, les dimensions d’une tableau 2D est représenté en (ligne,colonne).
\begin{lstlisting}
Dim tableau2D(5, 5) As Integer
Dim AutreTableau2D(1 To 5, 2 To 4) As Double
\end{lstlisting}        
        \end{itemize}
    \end{frame}
    %--- Next Frame ---%
    \subsection{Utilisation des cases du tableau}
    \begin{frame}[fragile]{Utilisation des cases du tableau}
        \begin{itemize}
            \item On invoque la case d’un tableau en utilisant un indice entre parenthèses.
            \item On utilise cette méthode pour assigner de nouvelle valeur ou aller consulter les valeurs déjà présentes.
            \item Les tableaux de plusieurs dimensions demandes le bon nombres d’indice (1 pour chaque dimensions).
\begin{lstlisting}
monTableau(3) = 4
If monTableau(3) = 4 Then
'Autre instructions

AutreTableau2D(2,3) = 2.43
\end{lstlisting}  
        \end{itemize}
    \end{frame}
    %--- Next Frame ---%
    \subsection{Allocation dynamique}
    \begin{frame}{Allocation dynamique}
        \begin{itemize}
            \item Un tableau dynamique ne contient pas de taille à sa définition.
            \item Il faut ensuite le redimensionner avant de l’utiliser avec Redim.
            \item On peut ensuite le redimensionner à volonté.
            \item On redimensionne avec le mot clé Preserve pour ne pas perdre l’information déjà contenu dans le tableau.
        \end{itemize}
    \end{frame}
    %--- Next Frame ---%
    \subsection{Passage de tableaux}
    \begin{frame}{Passage de tableaux}
        \begin{itemize}
            \item On passe des tableaux en paramètre sans indiquer leur taille. On ne fait qu’ajouter les parenthèses après l’identificateur pour indiquer que la variable sera un tableau.
            \item Les tableaux sont obligatoirement passés par référence. Il faut donc faire attention aux modifications de tableaux durant l’exécution de sous-programme.
            \item Pour savoir la taille d’un tableau, les fonction LBound et UBound permettent de déterminer respectivement l’indice inférieur et supérieur d’un tableau.
        \end{itemize}
    \end{frame}
    %--- Next Frame ---%
    \subsection{Retour de tableaux}
    \begin{frame}[fragile]{Retour de tableaux}
    Pour retourner un tableau, trois éléments doivent être mis en places:
        \begin{itemize}
            \item Le type de retour doit indiquer que le retour est un tableau (en ajoutant des parenthèses).
            \item Créez un tableau de retour qui va contenir le tableau qu’on veut retourner (on ne peux pas utiliser la variable de retour comme un tableau directement).
            \item Assignez la tableau de retour à la variable de retour.
        \end{itemize}
\begin{lstlisting}
Function retourTableau() As Integer()
    Dim retour() As Integer
    ReDim retour(5)
    retourTableau = retour
End Function
\end{lstlisting} 
    \end{frame}
    %--- Next Frame ---%
    \subsection{Destruction de tableaux dynamique}
    \begin{frame}[fragile]{Destruction de tableaux dynamique}
        \begin{itemize}
            \item Le mot clé \alert{Erase} permet de détruire un tableau dynamique déjà existant.
            \item Faire attention, Ubound/Lbound sur un tableau vide génère une erreur.
        \end{itemize}
\begin{lstlisting}
Sub TestErase()
    Dim tab() As Integer
    ReDim tab(5)
    tab(3) = 5
    
    Erase tab
End Sub
\end{lstlisting} 
    \end{frame}
    %--- Next Frame ---%
\end{document}