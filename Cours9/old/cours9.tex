\documentclass[aspectratio=169,usenames,dvipsnames]{beamer}
\usepackage{listings}

\usetheme[progressbar=frametitle]{metropolis}

\makeatletter
\setlength{\metropolis@titleseparator@linewidth}{2pt}
\setlength{\metropolis@progressonsectionpage@linewidth}{2pt}
\setlength{\metropolis@progressinheadfoot@linewidth}{2pt}
\makeatother

\lstset{language=[Visual]Basic, basicstyle=\footnotesize\ttfamily, keywordstyle=\color{blue}, commentstyle=\color{OliveGreen},frame=single}

\title{INF130 : Ordinateurs et programmation}
\subtitle{Semaine 9 – Classeurs et environnement d'Excel}
\author{Hugo Leblanc}

\begin{document}
    \maketitle

    %--- Next Frame ---%
    \begin{frame}{Sommaire}
        \tableofcontents
    \end{frame}
    %--- Next Frame ---%

    \section{Evironnement d'Excel}
    \begin{frame}{Environnement d’Excel}
        \begin{itemize}
            \item L’environnement d’Excel est un hiérarichie d’objets.
            \item Au plus haut niveau, nous avons le classeur (.xlsm).
            \item Le classeur contient des feuilles de calculs, des modules et des formulaires.
            \item Chaque feuille de calculs contiennent des cellules.
            \item Une plage est une collection de plusieurs cellules.
        \end{itemize}
    \end{frame}
    %--- Next Frame ---%
    \section{Contrôles de formulaire}
    \begin{frame}{Contrôles de formulaire}
        \begin{itemize}
            \item On doit avoir accès à l’onglet Développeur (on l’ajoute dans les options d’Excel) pour avoir accès au contrôles.
            \item Les contrôles peuvents être ajoutés à des feuilles de calcul ou des formulaires
        \end{itemize}
    \end{frame}
    %--- Next Frame ---%
    \begin{frame}{Contrôles de formulaire}
        \begin{itemize}
            \item Les cases à cocher permettent d'indiquer si une option est activée au moyen d'un crochet.
            \item Les cases d'options permettent de choisir une option parmi un groupe d'options mutuellement exclusives. Pour les utiliser, il faut tout d'abord ajouter un groupe d'options.
            \item Les compteurs et les barres de défilement permettent d'augmenter ou de réduire une valeur affichée.
            \item Les zones de liste et les zones combinées déroulantes permettent de choisir un élément dans une liste.
            \item Les boutons permettent l'exécution de macros ou de programmes VBA.
        \end{itemize}
    \end{frame}
    %--- Next Frame ---%
    \section{Les objets d'Excel}
    \begin{frame}{Les objets d'Excel}
        \begin{itemize}
            \item Un objet est un entité qui contient à la fois des propriétés (variables) et des méthodes (sous-programmes).
            \item Des centaines d’objets sont diponible dans Excel mais les plus important sont les suivants:
            \begin{itemize}
                \item Application – l’application d’Excel (et tous les classeurs ouvert)
                \item Workbook – Un classeur avec ses feuilles
                \item Worksheet – Une feuille de calcul
                \item Range – Une sélection de cellule d’une feuille
            \end{itemize}
            \item Les objets peuvent être utilisé en tant que paramètre de sous-programmes.
        \end{itemize}
    \end{frame}
    %--- Next Frame ---%
    \begin{frame}{Accéder au contenu d’une feuille}
        \begin{itemize}
            \item Il est possible d'accéder au contenu des cellules dans une feuille à l'aide des propriétés Range et Cells.
            \item Cells(ligne ,colonne) permet de faire réréfence à une cellule particulière.
            \item Range permet de sélectionner un plage de cellule voici quelque exemple d’utilisation:
            \begin{itemize}
            \item Range("A1")
            \item Range("A1:E5")
            \item Ragne("CELLULE\_NOMME")
            \item Range(Cells(2, 2), Cells(10, 5))
            \end{itemize}
        \end{itemize}

    \end{frame}
\end{document}